\documentclass[11pt]{article}
\usepackage{amsfonts,amsmath,amssymb,amsthm}
\usepackage{array}
\usepackage{bm}
\usepackage{commath}
\usepackage{enumitem}
\usepackage{fancyhdr,fancyvrb}
\usepackage[letterpaper,text={7in,9in},centering]{geometry}
\usepackage[pdftex]{graphicx}
\usepackage{ifthen}
\usepackage{mathrsfs}
\usepackage{multicol}
\usepackage{pifont}
\usepackage{sectsty}
\usepackage{setspace}
\usepackage{stackrel}
\usepackage{stmaryrd}
\usepackage{tensor}
\usepackage{xspace}
\usepackage{natbib}

\pagestyle{fancy}
\fancyhead[RO]{Edward Gan \& Max Wang}

\sectionfont{\large}
\subsectionfont{\normalsize}

\newcommand{\Sequitur}{\textsc{Sequitur}\xspace}

\begin{document}
\bibliographystyle{plain}

\title{Proposal: Loose Grammar Inference for Lossy Compression}
\author{Edward Gan \& Max Wang}
\date{October 17, 2012}
\maketitle

\section{Introduction and Goals}

\section{Background}

Though we have not found very much previous work in the area of lossy grammar
based compression, our proposed research will try to build upon powerful
results in grammar based compression and incorporate some of ideas currently
used in lossy compression algorithms.

\subsection{Grammar Compression}

As described in the introduction, the proposed research shares many of its
goals with the \Sequitur algorithm described in \cite{sequitur}.  The authors
of \cite{sequitur} demonstrate how context free grammars offer a means of
simultaneously revealing the structure of data streams and compressing them
efficiently.  They also develop an effective algorithm based on iteratively
rewriting grammars to keep them small and efficient. \Sequitur yields
intriguing grammar based analyses of texts and musical scores, inspection of
its results finds that it unable to detect and compress larger scale structure
because of small fluctuations that occur in natural streams.  This was part of
our inspiration for trying to achieve both better compression and better
structural analysis by ignoring small scale fluctuations.

In \cite{sequitur2}, the detailed compression scheme used in the \Sequitur
algorithm is described.  By sending the grammar over in an implicit,
local-pointer based scheme, they are able to represent the grammar with very
little overhead.  A similar scheme should be easily adaptable to any grammar
based compression scheme.

Finally, within the realm of \Sequitur based compression, \cite{nevillphd}
explores a variety of ad-hoc extensions to the \Sequitur algorithm which improve
its compression performance on structured data.  These ranged from introducing
domain-specific constraints to their grammar, adding a few steps of
backtracking to their normally greedy grammar formation, to guessing
unifications in attempting to infer recursive grammar rules.  Though these may
invalidate some of their theoretical results on the asymptotic performance of
\Sequitur, in practice they seem to have worked and add nicely to the grammar
inference framework.

The intution that grammar inference can support many detailed policies is made
more formal in \cite{grammarcodes}.  The authors classify the properties a
grammar needs to function as a good compressor, and moreover give a set of
reduction rules for putting a grammar into the appropriate form.  Within this
context, grammar inference algorithms similar to \Sequitur can be described as
simple applications of their reduction rules to different ways of generating
base grammars.  Even more examples of the variety of grammar inference schemes
possible under this general framework of reducing grammars can be found in
\cite{efficientgreedy}

There has also been very interesting work analyzing how grammar based
algorithms compare with the theoretical best lossless grammar for a string, in
the line of \cite{approximation}.  However, these analytic bounds for lossless
grammars are outside the scope of our proposed research.

In summary, established work on grammar inference algorithms provides a solid
set of tools for experimenting with the kinds of local modifications we propose
to make to improve grammar structure by introducing lossiness.

\subsection{Lossy Compression}

\section{Research Plan}

The proposed research project will involve designing new lossy grammar
compression algorithms, implementing them, and comparing their performance with
existing methods.  If time permits we would also like to prove lower bounds on
the runtime of our algorithm and upper bounds on the size of the grammar it
generates.

\subsection{Algorithm Design}

Starting with an existing grammar inference algorithm, or perhaps with a simple
variation based on the reduction rules in \cite{grammarcodes}, there are three
places we might introduce lossiness into the system.  We could
\begin{itemize}
  \item perform text-transforms on the input before applying a grammar
    inference algorithms;
  \item modify the grammar inference algorithm so that it generates a lossy
    grammar on the fly; or
  \item take a lossless grammar code and apply lossy transforms to it.
\end{itemize}
Of these, the first could provide the biggest potential for taking advantage of
source-specific knowledge of what kinds of data we can ignore.  The second
allows the most control over precisely how a grammar can be formed, and would
not be too difficult, we expect, to build into reduction-rule based frameworks.
The third would be the most general, since it could be applied to any CFG, but
may be difficult to get consistent performance from.

After a lossy CFG is obtained, we plan on using standard methods for encoding
the CFG such as those described in \cite{sequitur2}

\subsection{Evaluation Methodology}

There are two important metrics we plan on evaluating our algorithm with. First,
the lossiness it introduces should allow it to compress very well. This can be
quantified by comparing its compression ratio with other grammar based and 
general compression schemes on text sources, especially semi-structured text 
which might best exhibit nearly hierarchical data.

We would also like to make use of our algorithm to reveal the inherent 
structure in text data. Our hope is that allowing lossiness will alow it 
to capture larger scale structures, so we plan on subjectively evaluating 
the CFGs produced for how well they group the components that a human
might perceive in pieces of text.

Comparing our runtime with existing algorithms can be done in a
straightforward way.

We plan on evaluating our algorithm on standard english prose texts, but
we also suspect that the texts which will prove most interesting to 
analyze are those with slightly more structure imposed on them. Human-made
tables such as the genealogical trees discussed in \cite{nevillphd} are a
good example, as are simple musical pieces. Slightly more artificial
examples we might try include strings generated from actual CFGs but with
small amounts of noise added in.

\subsection{Implementation and Resources}

\subsection{Possible Issues}

\nocite{*}

\bibliography{sources}

\end{document}
